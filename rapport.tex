%%%%%%%%%%%%%%%%%%%%%%%%%%%%%%%%%%%%%%%%%%%%%%%%%%%
%%
%%  KEDJI Komlan Akpédjé <eric.kedji@gmail.com>
%%  http://erickedji.wordpress.com/
%%  
%%  Rapport du project de compilation (PseudoC)
%%  Jeudi 17 Avril 2008
%%  ENSIAS
%%
%%%%%%%%%%%%%%%%%%%%%%%%%%%%%%%%%%%%%%%%%%%%%%%%%%%

\documentclass[12pt,a4paper,openright]{report}
\usepackage[utf8]{inputenc}
\usepackage[english,french]{babel}
\usepackage{url}
\usepackage[T1]{fontenc}
\usepackage{charter}
\usepackage{verbatim} % for verbatim input
\usepackage{fancyhdr}
\usepackage{fancyvrb}
\usepackage[Sonny]{fncychap}
\usepackage{simplemargins}
\usepackage[pdftex]{graphicx}

% eviter l'espace après les deux points dans une url
\let\urlorig\url
\renewcommand{\url}[1]{%
   \begin{otherlanguage}{english}\urlorig{#1}\end{otherlanguage}%
}

\pagestyle{fancy}
% with this we ensure that the chapter and section
% headings are in lowercase.
\renewcommand{\chaptermark}[1]{\markboth{#1}{}}
\renewcommand{\sectionmark}[1]{\markright{\thesection\ #1}}
\fancyhf{} % delete current setting for header and footer
\fancyhead[LO]{\it\bfseries\leftmark}
\fancyhead[RO]{\bfseries\thepage}
\renewcommand{\headrulewidth}{0.5pt}
\renewcommand{\footrulewidth}{0pt}
\addtolength{\headheight}{0.5pt} % make space for the rule
\fancypagestyle{plain}{%
   \fancyhead{} % get rid of headers on plain pages
   \renewcommand{\headrulewidth}{0pt} % and the line
}

\newcommand{\nom}[1]{\textsc{\texttt{#1}}}
\newcommand{\pc}{\nom{PseudoC}}
\newcommand{\pcc}{\nom{PseudoC Compiler Collection}}
\newcommand{\ce}{\nom{C}}
\newcommand{\flex}{\nom{Flex}}
\newcommand{\bison}{\nom{Bison}}
\newcommand{\ensias}{\nom{ENSIAS}}

%%%%%%%%%%%%%%%%%%%%%%%%%%%%%%%%%%%%%%%%%%%% begin nss style title
\newlength{\centeroffset}
\setlength{\centeroffset}{-0.5\oddsidemargin}
\addtolength{\centeroffset}{0.5\evensidemargin}
\newcommand{\showtitle}{
    %\addtolength{\textwidth}{-\centeroffset}
    \thispagestyle{empty}
    \vspace*{\stretch{1}}
    \noindent\hspace*{\centeroffset}\makebox[0pt][l]{\begin{minipage}{\textwidth}
    \flushright
    {\Huge\bfseries Implémentation d'un générateur \\
    de byte-code et d'un interprêtateur \\
    pour le langage PseudoC
    }
    \noindent\rule[-1ex]{\textwidth}{5pt}\\[2.5ex]
    \hfill\emph{\Large Projet du module ``Compilation'' - \ensias{}}
    \end{minipage}}

    \vspace{\stretch{1}}
    \noindent\hspace*{\centeroffset}\makebox[0pt][l]{\begin{minipage}{\textwidth}
    \flushright
    {\bfseries 
    Komlan Akpédjé KEDJI\\[1.5ex]}
    Mars - Avril 2008 \\[1.5ex]
    Sous la direction de: {\bfseries Professeur Habilité Karim BAINA}
    \end{minipage}}
} % end new command showtitle
%\addtolength{\textwidth}{\centeroffset}
\vspace{\stretch{2}}

%%%%%%%%%%%%%%%%%%%%%%%%%%%%%%%%%%%%%% end nss style title

\setleftmargin{3cm}
\setrightmargin{3cm}
\settopmargin{3cm}
\setbottommargin{3cm}
\setlength{\parskip}{0.5cm}

\usepackage[pdftex,%
    colorlinks=true,%
    linkcolor=blue,%
    bookmarks=true,%
    pdftitle={Générateur de byte-code et Interpréteur pour le langage PseudoC - KEDJI Komlan Akpédjé},%
    pdfauthor={KEDJI Komlan Akpédjé (eric.kedji@gmail.com)}]{hyperref} %must be the last in the preambule!

\begin{document}

\showtitle

\tableofcontents

\begin{abstract}
Dans le cadre du cours de compilation, il est implémenté une collection d'outils
(\nom{PCC}: \pcc) permettant d'exécuter du code écrit en \pc{} (une version
allégée du langage \ce). Le présent document documente les diverses stratégies
et compromis utilisés tout au long du processus, et discute de leur pertinence
vu les résultats obtenus.
\end{abstract}



\chapter{Introduction}
    \begin{quote}
    \emph{``Fools ignore complexity. Pragmatists suffer it. Some can
    avoid it. Geniuses remove it.''} \\
    -- Alan J. Perlis (Epigrams in programming).\cite{EIP}
    \end{quote}

    S'il existe un trait caractéristique du développement d'un compilateur, il
    s'agit bien de la complexité\footnote{Il n'est donc pas surprennant que
    d'importants travaux de recherche soient menés sur le sujet, voir
    \cite{EHC}.}.Le présent rapport présente les stratégies utilisées pour
    contrôler cette complexité, dans le cadre de l'implémentation d'une suite
    d'outils permettant d'exécuter du code \pc.

    De \cite{EHC} nous retenons qu'une stratégie éfficace pour contrôler
    la complexité d'un compilateur est d'utiliser des représentations
    intermédiares bien définies. Ceci permet de décrire la transformation
    du program source en exécutable comme une série de transformations
    sur un flux de données. Le présent rapport est structuré de manière
    à réfléter ce processus.

    En effet, le passage d'un code source en \pc{} à l'exécution traverse les
    étapes suivantes:

    \emph{Code Source (PseudoC) -> Code Intermédiare -> Byte-Code -> Exécution}

\chapter{L'analyseur lexical de code \pc}
    L'analyseur lexical a été généré par \flex, à partir du fichier
    d'entrée \texttt{pc\_lexer.lex}.

    En dehors du code classique d'analyse lexical, les fonctionnalités
    suivantes sont introduites:

    \begin{itemize}
        \item Sauvegarde des positions (ligne, colonne) de chaque
        token. Ceci permettra par la suite de donner des messages
        d'erreur utiles lors de la compilation.
        \item Vérifications de base sur les chaines de caractères: les
        chaines invalides (contenant des retour chariot) et celles trop
        longues\footnote{Les chaines littérales en \pc{} ont une longueur
        maximale de 256 caractères.} sont détectées.
    \end{itemize}

\chapter{L'analyseur syntaxique de code \pc}
    L'analyseur lexical a été généré par \bison, à partir du fichier
    d'entrée \texttt{pc\_parser.y}.

    Le fichier \texttt{pc\_parser.y} implémente la grammaire du \pc. Le
    compilateur construit étant à une passe, l'AST (Abstract Syntax
    Tree) est implicite, et les actions sémantiques effectuent
    directement les appels nécessaires pour générer le code
    intermédiare. En effet, la construction d'un AST ne se justifie pas
    vu la simplicité des structures syntaxiques du \pc.

\begin{quote}
\emph{``There are two ways of constructing a software design; one way is to make
it so simple that there are obviously no deficiencies, and the other way
is to make it so complicated that there are no obvious deficiencies. The
first method is far more difficult.''} \\
-- C. A. R. Hoare
\end{quote}

    Une règle de la grammaire a donc la forme suivante:
\begin{verbatim}
instruction: WRITESTR '(' STR ')' ';'
    { emit(OPWRITESTR, $3); }
    ...
\end{verbatim}

    Ici, le token \texttt{WRITESTR} est reconnu, suivi d'une parenthèse
    ouvrante, un argument (une chaine de caractère littérale), puis une
    parenthèse fermante. L'action correspondante est de générer l'instruction de
    code intermédiaire nommée \texttt{OPWRITESTR}, avec comme argument, le
    troisième token, ie \texttt{STR}.
    
    Les actions sémantiques font des appels au module implémentant la
    table de symboles (\texttt{pc\_symtable.c}). Cette dernière est
    implémentée comme une table de hachage\footnote{L'implémentation de
    la table de hashage est contenue dans le fichier \texttt{uthash.h},
    et provient du projet \nom{UTHASH} sur
    \texttt{http://uthash.sourceforge.net}.}. Ceci permet de compiler de
    larges programmes \pc, sans pénalité de performance majeure (causée
    par une recherche linéaire dans une liste chainée).
    
\chapter{Le générateur de code intermédiare}
    Le fichier \texttt{ic\_generator.c} implémente la génération de code
    intermédiare. Le point central est la fonction \texttt{emit} dont le
    prototype est:

    \texttt{void emit(int instruction, ...);}

    \texttt{emit} prend un nombre variable d'arguments (suivant
    l'instruction) et génère le code approprié. Pour rendre les
    modifications plus simples, la correspondance entre le numéro d'une
    instruction du code intermédiare, et la représentation textuelle de
    l'opérateur, est maintenue dans le fichier \texttt{ic\_generator.h}
    (tous les modules désirant faire cette conversion y font appel).
    Ceci permet de modifier la représentation textuelle des opérateurs
    en mettant à jour un seul fichier.

    Le code intermédiare utilisé est à zero adresse. Le code généré
    étant destiné à être interprêté, il est tout à fait inutile
    d'utiliser un code à plus d'une adresse. En effet, d'une part les
    opérandes ne seront jamais rangés dans les régistres réels de la
    machine\footnote{Suggérer au compilateur C de mettre une variable
    dans un registre n'est pas suffisant, le standard \ce{} précise qu'il
    s'agit seulement d'une suggestion, pas un ordre.}, i.e. il n'y a
    aucun gain en performance. D'autre part, on évite ainsi la
    complexité liée à l'allocation des registres, et on bénéficie de la
    densité de code caractéristique des codes à une adresse.

\begin{quote}
\emph{``Any fool can make the simple complex, only a smart person can make the
complex simple.''} \\
-- [Anonymous]
\end{quote}

    Il faut noter que le code intermédiare choisi est optimisé pour la
    lisibilité. Il est possible de programmer directement dans le code
    intermédiare. le fichier généré à l'extension \texttt{.s}, ce qui permet de
    l'analyser avec la coloration syntaxique sous un éditeur comme \texttt{vim}.
    Ceci est d'une aide précieuse lors du debuggage.

\chapter{L'analyseur syntaxique de code intermédiare}
    Le code intermédiare généré étant du texte, il nous faut un analyseur
    synthaxique pour l'exploiter. Cette fonction est implémenté par
    le fichier \texttt{ic\_parser.c}

    Il s'agit de la première phase de l'interprétation. C'est une étape
    simple qui n'est isolée que pour des raisons de modularité.
    \texttt{ic\_parser.c} se résume à la fonction
    \texttt{next\_ic\_inst} qui retourne, à chaque appel, une
    représentation structurée de la prochaine instruction du fichier
    contenant le code intermédiaire. Il s'agit de récupérer une instruction du
    langage intermédiaire, et de le transformer en une structure de données
    directement exploitable par le générateur de byte-code.

\chapter{Le générateur de byte-code}
    Le fichier \texttt{bc\_generator.c} implémente le générateur de byte-code.

    Le byte-code généré est rangé dans un tableau \texttt{code} qui
    émule la RAM de l'ordinateur. Une copie est stockée dans un fichier,
    pour éviter de regénérer le byte-code lors des éxécutions
    ultérieures\footnote{\nom{Python} utilise une tactique similaire.
    Avant d'interpréter un fichier \texttt{foo.py}, Python génère un
    fichier de byte-code \texttt{foo.pyc}. Le byte-code est utilisé lors
    des appels ultérieurs.}.

    Le byte-code présente l'avantage d'être très compact. Les instructions
    sans opérande (comme les opérateurs arithmétiques) n'occupent q'un
    octect. Celles qui opérent sur des emplacements mémoire (comme
    l'assignation de variable) sont suivis d'une adresse qui est un
    entier stocké sur quatre octects. La taille d'une instruction varie
    donc entre 1 et 5 octets.

\chapter{L'interpréteur}
    Le fichier \texttt{interpreter.c} implémente l'interprétation du
    byte-code, et a contient essentiellement la fonction \texttt{run}.
    Il faut noter que la convention d'appel (ou \emph{stack-layout}) implémentée
    ici pour le \pc{} est la même que celle du langage \ce.

    Au moment de l'éxécution, \texttt{run} utilise les structures
    suivantes:

    \begin{itemize}
        \item Une pile d'appels (\texttt{call\_stack}). Elle contient
        les adresses de retour des fonctions et les copies des adresses
        des cadres d'activation (voir plus bas).
        \item Un pointeur de pile d'appels (\texttt{call\_stackp}). Il
        pointe sur le prochain emplacement libre sur la pile d'appels.
        \item Une pile d'expressions (\texttt{expr\_stack}). Elle
        contient les valeurs sur lequelles les opérateurs calculent
        (le code est à une adresse: tous les calculs sont faits sur la
        pile).
        \item Un pointeur de pile d'expressions (\texttt{expr\_stackp}).
        Il pointe sur le prochain emplacement libre sur la pile
        d'expressions.
        \item Un pointer de frame (\texttt{framep}). Il contient
        l'adresse du début du cadre d'activation courant (i.e. celui de la
        fonction en cours d'exécution).
        \item Un compteur de programme (\texttt{instp}). Il contient
        l'adresse de l'instruction suivante.
    \end{itemize}

\chapter{Utilitaires et support de debuggage}
    Une série d'utilitaires a été mise au point pour faciliter
    l'utilisation de \pcc, et surtout faciliter le debuggage.

    Commençons par noter que l'analyseur syntaxique est par défaut très
    verbeux. Il donne des détails précis sur les erreurs rencontrées
    pour permettre au programmeur des les répérer et les corriger le
    plus vite possible. Un exemple de sortie est le suivant:

\begin{verbatim}
Duplicated identifier: `i': line 68, line 70
Type mismatch: variable `tableau':
    decl[ARRAY]: line 67, use[INT]: line 80
Function `longueur' used as a variable:
    decl: line 4, use: line 119
Redefined function: `afficher': line 8, line 186
Undeclared variable: `total': line 98
Function prototype mismatch: `size': line 1, line 2
    First: Function `size':       INT -> INT
    Second: Function `size':      [Procedure] -> INT
Undeclared function `foobar' (line 5)
`bar' is not a function (line 6)
\end{verbatim}

    \section{L'utilitaire \texttt{pseudocl}}
        \texttt{pseudocl} affiche une version \textit{human-readable}
        d'une source \pc. Il est ainsi possible de visualiser le
        résultat obtenu après la phase d'analyse lexicale. Exemple
        d'utilisation:

\begin{verbatim}
$ pseudocl <<EOF
> int foo; static int bar;
> int main(void){ write_int(2*5/6); return 0;}
> EOF
INT         IDENTIFIER  SEMI_COLON
STATIC      INT         IDENTIFIER  SEMI_COLON
INT         IDENTIFIER  PAREN_OPEN  VOID
PAREN_CLOSE BRACE_OPEN  WRITE_INT   PAREN_OPEN
NUMBER      MULTIPLY    NUMBER      DIVIDE
NUMBER      PAREN_CLOSE SEMI_COLON  RETURN
NUMBER      SEMI_COLON  BRACE_CLOSE
$ 
\end{verbatim}

    \section{L'utilitaire \texttt{opname}}
        Quand on ne dispose que du byte-code (on peut en avoir une
        représentation `lisible' avec l'utilitaire \nom{UNIX}{} \texttt{xxd}),
        il est utile de pouvoir obtenir le nom d'un opérateur à partir de
        son numéro donné sous forme décimale ou hexadécimale. L'utilitaire
        \texttt{opname} a été conçu à cet effet. Exemple d'utilisation:

\begin{verbatim}
$ opname
 Utility to get an operator name from its code
  -> code in decimal: no prefix (ex: `45')
  -> code in hexadecimal: prefix `x' (ex: `x1c')

        Usage:   opname <dec_code> | opname x<hex_code>

$ opname 30
 dec=30  hex=1e  CALL
$ opname x10
 dec=16  hex=10  POP
$ opname x41
 Unknown opcode: dec=65  hex=41
$
\end{verbatim}


    \section{Mode pas à pas dans l'interpréteur}
        L'interprétateur fonctionne en mode pas à pas, lorque la
        macro \nom{NDEBUG} n'est pas définie au moment de la compilation.
        
        Dans ce mode, la confirmation de l'utilisateur est demandée
        avant l'exécution de chaque instruction du byte-code, et de très
        utiles informations (état de la pile d'appel, état de la pile
        d'instruction, code et nom de l'instruction suivante, ...) sont
        affichées. Exemple d'éxécution:

\begin{verbatim}
 -> call stack dump:
 [7]: 2         (hex:2)
 [6]: 0         (hex:0)
 [5]: 0         (hex:0)
 [4]: 42        (hex:2a)
 [3]: 42        (hex:2a)
 [2]: 42        (hex:2a)
 [1]: 6         (hex:6)
 [0]: 0         (hex:0)
 -> expr stack dump:
 [0]: 2         (hex:2)
 next instruction: 38 (PUSHC)   at 212  (hex:d4)
 press enter to continue
\end{verbatim}

\chapter{Le jeu de tests}
    \section{Tests de syntaxe}
        Lors du développement, il est utile de pouvoir tester la
        validité de l'analyseur syntaxique. Il est cependant difficile
        d'écrire un programme de test contenant toutes les combinaisons
        syntaxiques possible.

        Un programe \nom{Scheme}, \texttt{generator.scm}, qui génère
        aléatoirement des programmes \pc{} syntaxiquement valides, a donc été
        implémenté. Une approche similaire a discutée dans \cite{AGTC}.

        Le programme \texttt{generator.scm} est lui-même une démonstration de la
        puissance méta-syntaxique des macros en \nom{Scheme}.
        En effet, le code ressemble à s'y méprendre à la spécification
        \nom{BNF} du \pc. Extrait:

\begin{verbatim}
(define-production (int)
  (number->string (random 1000000)))

(define-production (binary-op d)
  ">" "<" "==" "<=" ">=" "!=" "+"
  "-" "*" "/" "%" "||" "&&")

(define-production (function-body)
  (string-append "{\n"
      (list-of declaration "\n\t")
      "\n"
      (list-of instruction "\n\t")
      "\n}\n"))
\end{verbatim}

        Le script shell \texttt{syntaxtest.sh} permet de lancer un
        nombre paramétrable de tests sur des programmes
        générés avec \texttt{generator.scm}.

    \section{Tests de sémantique}
        Une élégant framework de test a été mis en place pour le compilateur
        \pc. Le framework s'appuie sur le constat qu'un programme \pc{} est à
        quelques modifications près, un programme \ce{} valide. On peut donc
        compiler un même programme avec le compilateur \pc{} (\texttt{pseudocc})
        et le compilateur \ce{} (\texttt{gcc}), et comparer les sorties des deux
        exécutables. Une approche similaire a été utilisée par \cite{FS} pour
        détecter plusieurs bugs dans \nom{GCC}.


        Le framework de test contient les fichiers suivants:

        \begin{itemize}
            \item \texttt{libpc.c}: implémentation en \ce{} des appels
            système du \pc{} (\texttt{write\_int}, \texttt{write\_string},
            \ldots).
            \item \texttt{libpc.h}: définitions de macro permettant
            de générer du code \ce{} ou du code \pc{} à partir d'une représentation
            unifiée.
            \item \texttt{test.h}: fonctions générales utiles
            pour les cas de test (valides en \ce{} et en \pc).
            \item \texttt{runtest.sh}: shell script permettant d'exécuter un
            test.
            \item \texttt{runall.sh}: shell script permettant d'exécuter tous
            les cas de test disponibles.
            \item \texttt{identical.c}: source de l'utilitaire permettant de
            détecter si deux fichiers sont identiques. Même si cet utilitaire
            donne l'impression de réinventer \texttt{diff}, il a l'avantage de
            terminer à la détection de la première différence entre les deux
            fichiers, ce qui est plus efficace dans notre cas.
        \end{itemize}

\begin{quote}
\emph{``The good thing about reinventing the wheel is that you get a round
one.''} \\
-- Douglas Crockford (Author of JSON and JsLint)
\end{quote}

        Les cas de test implémentés sont les suivants:
        \begin{description}
            \item[io]:\\ Opérations d'entrée sortie (write\_int,
            write\_string)
            \item[math]:\\ Calculs arithmétiques (+, -, /, *, \%)
            \item[array]:\\ Utilisation des tableaux
            \item[boolean]:\\ Test des opérateurs booléen (\&\&, ||, !, short-circuit)
            \item[flow]:\\ Opérateurs de controle (if, while)
            \item[classics]:\\ Fonctions classiques (factoriel, fibonnaci,
            ackermann, primalité, plus grand commun diviseur)
        \end{description}

        Bien entendu, le compilateur passe tous les tests avec succès.

\chapter{Conclusion}
\begin{quote}
\emph{``[\dots] Hence my urgent advice to all of you to reject the morals of the
bestseller society and to find, to start with, your reward in your own
fun. This is quite feasible, for the challenge of simplification is so
fascinating that, if we do our job properly, we shall have the greatest
fun in the world.''} \\
-- E. W. Dijkstra, On the nature of computing science.\cite{ONCS}
\end{quote}

    Le développement de \pcc{} a été une immense partie de plaisir.

    Le développement d'un compilateur est une occasion unique pour entrer en
    contact avec toutes les belles idées qui font le charme de notre métier,
    l'informatique. En effet, c'est l'occasion d'explorer tous les concepts,
    depuis les langages de haut-niveau (ici le \pc) jusqu'au byte-code. C'est
    aussi une démontration de l'efficacité qui peut résulter de la combinaison
    des outils en ligne de commande sous \nom{Unix} (le \texttt{shell},
    \texttt{xxd}, \texttt{make}, \texttt{vim}, \ldots)

    Je suis globalement satisfait du travail accompli, même s'il
    reste du travail à faire pour que le \pcc{} approche de \nom{GCC} en
    termes de convivialité d'utilisation. Je pense en particulier à:

    \begin{itemize}
        \item Intégrer du code pour un traitement décent des
        arguments de la ligne de commande avec \nom{getopts}.
        \item Inclure des scripts pour une installation dans le style
        \texttt{./configure; make; make install} (\nom{GNU Autotools}).
        \item Implémenter l'optimisation du code intermédiaire généré
        (notament le \emph{peep-hole optimization}).
        \item Intégrer proprement \nom{Cpp} dans le processus de
        compilation pour que les programmes écrits en PseudoC puissent
        utiliser des instructions du préprocesseur \ce.
        \item Documenter proprement le code.
    \end{itemize}

\begin{thebibliography}{99}
    \bibitem{EHC} Atze Dijkstra, Jeroen Fokker and S. Doaitse Swierstra:
    \emph{The Structure of the Essential Haskell Compiler or Coping with
    Compiler Complexity}, \\
    \url{http://www.cs.uu.nl/wiki/Ehc/WebHome/}

    \bibitem{ONCS} E. W. Dijkstra, \emph{On the nature of computing
    science}, \\
    \url{http://www.cs.utexas.edu/users/EWD/}

    \bibitem{EIP} Alan J. Perlis, \emph{Epigrams in programming}, 
    SIGPLAN Notices Vol. 17, No. 9, September 1982, pages 7 - 13
    \url{http://www.cs.yale.edu/quotes.html}

    \bibitem{FS} Flash Sheridan, \emph{Practical Testing of a C99
    Compiler Using Output Comparison}.

    \bibitem{AGTC} D. L. Bird, C. U. Munoz, \emph{Automatic generation
    of random self-checking test cases}.

    \bibitem{BOCD} Torben Ægidius Mogensen, \emph{Basics of Compiler
    Design}.

    \bibitem{CCYB} Tom Niemann, \emph{A compact Guide to LEX \& YACC}.

\end{thebibliography}
\end{document}
